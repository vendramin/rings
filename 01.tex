%\section{Lecture -- Week 1}


\section{Rings}

The objective of the next five lectures is to extract certain properties inherent to integers and adapt them for broader applications. Initially, we will identify similarities between the ring of integers and the ring of real polynomials, 
followed by a more in-depth exploration of these specific attributes in more general contexts. A pivotal similarity between integers and real polynomials lies in the existence of the division algorithm. Nevertheless, it is essential to emphasize that $\Z$ and $\R[X]$ are fundamentally distinct entities, like apples and oranges. For instance, in $\R[X]$, 
the presence of the variable $X$ allows us to utilize formal derivatives 
(with respect to $X$), a feature absent in the ring of integers. 


\begin{definition}
\index{Ring}
A \emph{ring} is a non-empty set $R$ with two binary operations, the addition
$R\times R\to R$, $(x,y)\mapsto x+y$, and the multiplication
$R\times R\to R$, $(x,y)\mapsto xy$, such that
the following properties hold:
\begin{enumerate}
    \item $(R,+)$ is an abelian group.
    \item $(xy)z=x(yz)$ for all $x,y,z\in R$.
    \item $x(y+z)=xy+xz$ for all $x,y,z\in R$.
    \item $(x+y)z=xz+yz$ for all $x,y,z\in R$.
    \item There exists $1_R\in R$ such that $x1_R=1_Rx=x$ for all $x\in R$.
\end{enumerate}
\end{definition}

Our definition of a ring is that of a ring with identity. In general one
writes the identity element $1_R$ as $1$ if there is no risk of confusion.

\begin{definition}
\index{Ring!commutative}
A ring $R$ is said to be \emph{commutative} if $xy=yx$ for all $x,y\in R$. 
\end{definition}

\begin{example}
$\Z$, $\Q$, $\R$, and $\C$ are commutative rings.
\end{example}

Most of the students will have some familiarity with
real polynomials from their school days. 

\begin{example}
\index{Degree of a polynomial}
    The set  
    \[
		\R[X]=\left\{\sum_{i=0}^na_iX^i:n\in\Z_{\geq0},\,a_0,\dots,a_n\in \R\right\}
    \]
    of real polynomials in one variable 
    is a commutative ring with the usual operations.  
    For example, if $f(X)=1+5X^3$ and $g(X)=3X-2X^3$, then
    \begin{align*}
        f(X)+g(X) &= 1+3X+3X^3,\\
        f(X)g(X) &= 3X-2X^3+15X^4-10X^6.
    \end{align*}
    Recall that, if 
    \[
    f(X)=a_0+a_1X+a_2X^2+\cdots+a_nX^n
    \]
    and $a_n\ne0$, then $a_n$ is the \emph{leading coefficient} of $f(X)$ and 
    $n$ is the \emph{degree} of $f(X)$. In this case, we use the notation 
    $n=\deg f(X)$. 
\end{example}

More generally, if $R$ is a commutative ring, then $R[X]$ is a commutative ring. This construction
allows us to define 
the polynomial ring $R[X,Y]$ in two commuting variables $X$ and $Y$ and coefficients in $R$ as 
$R[X,Y]=(R[X])[Y]$. One can also define the ring  
$R[X_1,\dots,X_n]$ of polynomials 
in $n$ commuting variables $X_1,\dots,X_n$ with coefficients in $R$ as 
\[
R[X_1,\dots,X_n]
=(R[X_1,\dots,X_{n-1}])[X_n].
\]

\begin{example}
    If $A$ is an abelian group, then 
    the set 
    $\End(A)$ of group homomorphisms $A\to A$ is a ring with
    \[
    (f+g)(x)=f(x)+g(x),\quad
    (fg)(x)=f(g(x)),\quad f,g\in\End(A)\text{ and }x\in A.
    \]
\end{example}

If $X$ is a set, we write $|X|$ to denote 
the size of $X$. 

\begin{exercise}
    \label{xca:basic_formulas}
Let $R$ be a ring. Prove the following facts: 
\begin{enumerate}
    \item $x0=0x=0$ for all $x\in R$.
    \item $x(-y)=-xy$ for all $x,y\in R$.
    \item If $1=0$, then $|R|=1$. 
\end{enumerate}
\end{exercise}

\begin{example}
    The real vector space $H(\R)=\{a1+bi+cj+dk:a,b,c,d\in\R\}$ with basis $\{1,i,j,k\}$ 
    is a ring with the multiplication induced by
    the formulas 
    \[
    i^2=j^2=k^2=-1,
    \quad ij=k,
    \quad jk=i,
    \quad ki=j.
    \]
    As an example, let us perform a calculation in $H(\R)$: 
    \[
    (1+i+j)(i+k)=i+k-1+ik+ji+jk=i+k-1-j-k+i=-1+2i-j,
    \]
    as $ik=i(ij)=-j$ and $ji=-k$. This is the ring of real \emph{quaternions}.
\end{example}

\begin{example}
    Let $n\geq2$. 
    The abelian group $\Z/n=\{0,1,\dots,n-1\}$ of integers modulo $n$ is a ring 
    with the usual multiplication modulo $n$. 
\end{example}

\begin{example}
    Let $n\geq1$. 
    The set $M_n(\R)$ of real $n\times n$ matrices is a ring with the usual matrix operations. Recall
    that if $a=(a_{ij})$ and $b=(b_{ij})$, the multiplication $ab$ is given by
    \[
    (ab)_{ij}=\sum_{k=1}^n a_{ik}b_{kj}.
    \]
    Note that $M_n(\R)$ is conmutative if and only if $n=1$. 
\end{example}

Similarly, for any ring $R$ one defines the ring $M_n(R)$ of $n\times n$ matrices
with coefficients in $R$. 

\begin{example}
\label{exa:direct_rings}
    \index{Product direct of rings}
    Let $R$ and $S$ be rings. Then 
    $R\times S=(\{(r,s):r\in R,s\in S\}$ 
    is a ring with the operations 
    \[
    (r,s)+(r_1,s_1)=(r+r_1,s+s_1),\quad 
    (r,s)(r_1,s_1)=(rr_1,ss_1).
    \]
    The zero of $R\times S$ is $(0_R,0_S)$ and 
    the unit element of $R\times S$ is $(1_R,1_S)$. The ring 
    $R\times S$ is known as the \emph{direct product} of $R$ and $S$. 
\end{example}

\begin{definition}
\index{Subring}
    Let $R$ be a ring. A \emph{subring} $S$ of $R$ is a subset $S$ such that
    $(S,+)$ is a subgroup of $(R,+)$ such that $1\in S$ and 
    if $x,y\in S$, then $xy\in S$. 
\end{definition}

For example, $\Z\subseteq\Q\subseteq\R\subseteq\C$ is a chain of subrings. 

\begin{example}
    \index{Gauss integers}
    $\Z[i]=\{a+bi:a,b\in\Z\}$ is a subring of $\C$. 
    This is known as the ring of \emph{Gauss integers}.  
\end{example}

A \emph{square-free} integer is an integer that is divisible by 
no perfect square other than 1. Some examples: 2, 3, 5, 6, 7, and 10. 
The numbers 4, 8, 9, and 12 are not square-free. 

\begin{example}
	Let $N$ be a square-free integer. Then $\Z[\sqrt{N}]$ is
	a subring of $\C$.  	
\end{example}

If $N$ is a square-free integer, then 
$a+b\sqrt{N}=c+d\sqrt{N}$ in $\Z[\sqrt{N}]$ 
if and only if $a=c$ and $b=d$. Why?

\begin{example}
    $\Q[\sqrt{2}]=\{a+b\sqrt{2}:a,b\in\Q\}$ is a subring of $\R$. 
\end{example}

Why in the ring $\Q[\sqrt{2}]$ 
equality $a+b\sqrt{2}=c+d\sqrt{2}$ implies $a=c$ and $b=d$?

\begin{example}
    \index{Center!of a ring}
    If $R$ is a ring, then the \emph{center} 
    \[
    Z(R)=\{x\in R:xy=yx\text{ for all $y\in R$}\}
    \]
    is a subring of $R$. 
\end{example}

If $S$ is a subring of a ring $R$, then the zero-element 
of $S$ is the zero-element of $R$, i.e., $0_R=0_S$. Moreover, 
the additive inverse of an element $s\in S$ 
is the additive inverse of $s$ as an element of $R$. 	

\begin{exercise}\
    \label{xca:subrings}
\begin{enumerate}
	\item If $S$ and $T$ are subrings of $R$, then $S\cap T$ is a subring of $R$.
	\item If $R_1\subseteq R_2\subseteq\cdots$ is a sequence of subrings of $R$, then 
	$\cup_{i\geq1}R_i$ is a subring of $R$. 
    \item If $S$ and $T$ are subrings of $R$ such that $S\cup T$ is a subring, then $S\subseteq T$ or  $T\subseteq S$.
\end{enumerate}
\end{exercise}

\begin{definition}
\index{Units}
	Let $R$ be a ring. An element $x\in R$ is 
	a \emph{unit} if there exists $y\in R$ such that $xy=yx=1$. 
\end{definition}

The set $\mathcal{U}(R)$ of units of a ring $R$ form 
a group with the multiplication. For example, $\mathcal{U}(\Z)=\{-1,1\}$ and 
$\mathcal{U}(\Z/8)=\{1,3,5,7\}$. 

\begin{exercise}
\label{xca:units_R[X]}
Compute $\mathcal{U}(\R[X])$.
\end{exercise}

\begin{definition}
	\index{Ring!division}
	A \emph{division ring} is a ring $R\neq\{0\}$ 
	such that $\mathcal{U}(R)=R\setminus\{0\}$.  	
\end{definition}

\index{Quaternions}
The ring $H(\R)$ of \emph{real quaternions} is a non-commutative division ring. Find the inverse of
an arbitrary element $a1+bi+cj+dk\in H(\R)$. 

\begin{definition}
\index{Field}
	A \emph{field} is a commutative division ring (with $1\ne 0$). 
\end{definition}

For example, $\Q$, $\R$, and $\C$ are fields. 
If $p$ is a prime number, then $\Z/p$ is a field. 	

\begin{exercise}
\label{xca:Qsqrt2}
	Prove that $\Q[\sqrt{2}]$ is a field. 
	Find the multiplicative inverse of a non-zero element of the form 
	$x+y\sqrt{2}\in\Q[\sqrt{2}]$.  
\end{exercise}

More challenging: Prove that 
\[
\Q[\sqrt[3]{2}]=\{x+y\sqrt[3]{2}+z\sqrt[3]{4}:x,y,z\in\Q\}
\]
is a field. What is the inverse of a non-zero element of the form $x+y\sqrt[3]{2}+z\sqrt[3]{4}$?

\section{Ideals}

\begin{definition}
\index{Ideal!left}
	Let $R$ be a ring. A \emph{left ideal} of $R$ is a subset $I\subseteq R$ such that 
	$(I,+)$ is a subgroup of $(R,+)$ and $RI\subseteq I$, 
	i.e. $ry\in I$ for all $r\in R$ and $y\in I$. 
\end{definition}

\index{Ideal!right}
Similarly, one defines \emph{right ideals}, one needs 
to replace the condition $RI\subseteq I$ by 
the inclusion 
$IR\subseteq I$. 

\begin{example}
Let $R=M_{2}(\mathbb{R})$. Then   
\[
I=\left(\begin{array}{cc}
\mathbb{R} & \mathbb{R}\\
0 & 0
\end{array}\right)=\left\{ \left(\begin{array}{cc}
x & y\\
0 & 0
\end{array}\right):x,y\in\mathbb{R}\right\} 
\]
is a right ideal $R$ that is not a left ideal. 
\end{example}

Can you find an 
example of a left ideal that is not a right ideal?

\begin{definition}
\index{Ideal}	
Let $R$ be a ring. An \emph{ideal} of $R$ 
is a subset that is both a left and a right ideal of $R$. 
\end{definition}
 
If $R$ is a ring, then $\{0\}$ and $R$ are both ideals of $R$. 

\begin{exercise}
\label{xca:ideals}
Let $R$ be a ring. 
\begin{enumerate}
\item If $\{I_\alpha:\alpha\in\Lambda\}$ is a collection of ideals of $R$, then $\cap_{\alpha}I_\alpha$ is an ideal of $R$.  	
\item If $I_1\subseteq I_2\subseteq\cdots$ is a sequence of ideals of $R$, then $\cup_{i\geq1}I_i$ is an ideal of $R$. 
\end{enumerate}
\end{exercise}


\begin{example}
Let $R=\R[X]$. If $f(X)\in R$, then the set 
\[
(f(X))=\{f(X)g(X):g(X)\in R\}
\]
of multiples of $f(X)$ is an ideal of $R$. One can prove that this is the smallest 
ideal of $R$ containing $f(X)$.  	
\end{example}

If $R$ is a ring and $X$ is a subset of $R$, one defines
the ideal generated by $X$ as the smallest ideal of $R$ containing $X$, that is 
\[
(X)=\bigcap\{I:\text{$I$ ideal of $R$ such that $X\subseteq I$}\}.
\]
One proves that 
\[
(X)=\left\{\sum_{i=1}^mr_ix_is_i:m\in\Z_{\geq0},\,x_1,\dots,x_m\in X,\,r_1,\dots,r_m,s_1,\dots,s_m\in R\right\}, 
\]
where by convention, the empty sum is equal to zero. If $X=\{x_1,\dots,x_n\}$ is a finite
set, then we write $(X)=(x_1,\dots,x_n)$. 

\begin{exercise}
\label{xca:ideals_Z}
Prove that every ideal of $\Z$ is of the form $n\Z$ for some $n\geq0$. 	
\end{exercise}

\begin{exercise}
\label{xca:ideals_Zn}
Let $n\geq2$. Find the ideals of $\Z/n$. 	
\end{exercise}

\begin{exercise}
\label{xca:ideals_R}
Find the ideals of $\R$.	
\end{exercise}

A similar exercise is to find the ideals of any division ring.  

\begin{exercise}
\label{xca:ideals_matrices}
    Let $R$ be a commutative ring. Prove that every ideal of $M_n(R)$ is 
    of the form $M_n(I)$ for some ideal $I$ of $R$. 
\end{exercise}

%\begin{sol}{xca:ideals}
%    Recall the formula $E_{ij}E_{jk}=E_{ik}$ for elementary matrices.  
%    
%    Let $J$ be an ideal of $M_n(R)$. For a matrix $A=(a_{ij})$,
%    $E_{ki}AE_{jl}=a_{ij}E_{kl}$ for all $i,j,k,l$. Let 
%    \[
%    I=\{r\in R:rE_{11}\in J\}.
%    \]
%    Then $I$ is an ideal of $R$. Since $a_{ij}E_{11}=E_{1i}AE_{j1}$, 
%    $J\subseteq M_n(I)$. 
%    
%    Conversely, if $A\in M_n(I)$, then $A$ is a sum of elements 
%    of the form $rE_{ij}$ for $r\in I$. Thus $rE_{ij}\in J$ for $r\in I$ 
%    and hence $A\in J$. 
%\end{sol}


\begin{definition}
\index{Ideal!principal}
Let $R$ be a ring and $I$ be an ideal of $R$. Then $I$ is \emph{principal}
if $I=(x)$ for some $x\in R$. 	
\end{definition}

The division algorithm shows that every ideal of $\Z$ is principal, 
see Exercise~\ref{xca:ideals_Z}.  

The ring $\R[X]$ of real polynomials also has a division algorithm. Given the polynomials 
$f(X)\in\R[X]$ and $g(X)\in\R[X]$ with $g(X)\ne 0$, there are  
polynomials $q(X)\in\R[X]$ and $r(X)\in\R[X]$ such that 
\[
f(X)=q(X)g(X)+r(X),
\]
where $r(X)=0$ or $\deg r(X)<\deg g(X)$. 

For example, 
if $f(X)=2X^5-X$ and $g(X)=X^2+1$, then 
\[
f(X)=q(X)g(X)+r(X),
\]
where $q(X)=2X^3-2X$ and $r(X)=X$. 

\begin{exercise}
\label{xca:R[X]_principal}
	Prove that every ideal of $\R[X]$ is principal. 
\end{exercise}

If $K$ is a field, there is a division algorithm in the 
polynomial ring $K[X]$. Then one proves 
that every ideal of $K[X]$ is principal.  

\begin{exercise}
\label{xca:x_unit}
	Let $R$ be a commutative ring and $x\in R$. Prove that $x\in\mathcal{U}(R)$ if and only if
	$(x)=R$. What happens if $R$ is non-commutative?
\end{exercise}

A division ring (and, in particular, a field) has only two ideals. 

\section{Ring homomorphisms}

\begin{definition}
\index{Ring!homomorphism}
Let $R$ and $S$ be rings. A map $f\colon R\to S$ is a \emph{ring homomorphism}  
if $f(1)=1$, $f(x+y)=f(x)+f(y)$ and $f(xy)=f(x)f(y)$ for all $x,y\in R$. 	
\end{definition}

Our definition of a ring is that of a ring with identity. This means
that the identity element $1$ of a ring $R$ 
is part of the structure. For that reason, in 
the definition
of a ring homomorphism $f$ one needs $f(1)=1$.  

\begin{example}
The map $f\colon\Z/6\to\Z/6$, $x\mapsto 3x$, satisfies 
\[
f(xy)=f(x)f(y)\quad\text{and}\quad f(x+y)=f(x)+f(y)
\]
for all $x,y\in\Z/6$. However, $f$ is not a ring homomorphism because~$f(1)=3$. 	
\end{example}
 
If $R$ is a ring, then  
the identity map $\id\colon R\to R$, $x\mapsto x$, is a ring homomorphism. 	

\begin{example}
The inclusions $\Z\hookrightarrow\Q\hookrightarrow\R\hookrightarrow\C$ 
are ring homomorphisms. 	
\end{example}

More generally, if $S$ is a subring of a ring $R$, then the inclusion map 
$S\hookrightarrow R$ is a ring homomorphism. 

\begin{example}
Let $R$ be a ring. 
The map $\Z\to R$, $k\mapsto k1$, is a ring homomorphism. 	
\end{example}

\begin{example}
Let $x_0\in\R$. The evaluation map $\R[X]\to\R$, $f\mapsto f(x_0)$, 
is a ring homomorphism. 	
\end{example}

The \emph{kernel} of a ring homomorphism
$f\colon R\to S$ is the subset
\[
\ker f=\{x\in R:f(x)=0\}.
\]
One proves that the kernel of $f$ is an ideal of $R$.  
Moreover, recall from group theory that 
$\ker f=\{0\}$ if and only if $f$ is injective. The image 
\[
f(R)=\{f(x):x\in R\}
\]
is a subring of $S$. In general, $f(R)$ is not an ideal of $S$. 

\begin{example}
The inclusion $\Z\hookrightarrow\Z[X]$ is a ring homomorphism. Then
$2\Z$ is an ideal of $\Z$ but not an ideal of $\Z[X]$. 
\end{example}

\begin{example}
    Let $R$ and $S$ be rings and $R\times S$ be the direct product (see Example~\ref{exa:direct_rings}). 
    The map $R\to R\times S$, $r\mapsto (r,0)$ is an injective ring homomorphism. The 
    map 
    \[
    \pi_R\colon R\times S\to R,\quad (r,s)\mapsto r,
    \]
    is a surjective ring homomorphism with $\ker\pi_R=\{(0,s):s\in S\}=\{0\}\times S$. 
\end{example}

\begin{example}
	The map $\C\to M_2(\R)$, $a+bi\mapsto\begin{pmatrix}a&b\\-b&a\end{pmatrix}$, is an injective
	ring homomorphism. 	
\end{example}

\begin{example}
The map $\Z[i]\to\Z/5$, $a+bi\mapsto a+2b\bmod 5$, is a ring homomorphism 
with $\ker f=\{a+bi:a+2b\equiv 0\bmod 5\}$. 	
\end{example}

\begin{exercise}
\label{xca:Z6->Z15}
There is no ring homomorphism $\Z/6\to\Z/15$. Why?	
\end{exercise}

\begin{exercise}
\label{xca:evaluation_map}
If $f\colon\R[X]\to\R$ is a ring homomorphism 
such that the restriction $f|_{\R}$ of 
$f$ onto $\R$ is the identity, then there exists $x_0\in\R$ such that 
$f$ is the evaluation map at~$x_0$. 
\end{exercise}

