%\section{Lecture -- Week 7}
%\label{7}

\section{Maschke's theorem} 

Recall that, by convention, we only consider complex 
finite-dimensional representations of finite groups.

\begin{theorem}[Maschke]
\index{Maschke's theorem}
    Every representation of a finite group is completely reducible.
\end{theorem}

\begin{proof}
    Let $G$ be a finite group and $\rho\colon G\to\GL(V)$ be a representation of $G$. We proceed
    by induction on $\dim V$.
    If $\dim V=1$, the result is trivial, as degree-one representations are irreducible. Assume that
    the result holds for representations of degree $\leq n$. Suppose that $\rho$ has degree $n+1$. 
    If $\rho$ is irreducible, we are done. If not, use 
    Proposition \ref{pro:irr_or_dec} to 
    write $V=S\oplus T$, where $S$ and $T$
    are non-zero invariant subspaces of $V$. Since $\dim S<\dim V$ and $\dim T<\dim V$, it follows from
    the inductive hypothesis that
    both $S$ and $T$ are spaces of completely reducible representations. 
    Thus $\rho$ is completely reducible.
\end{proof}

\begin{example}
    Let $G=\Sym_3$ and $\rho\colon G\to\GL_3(\C)$ be the representation given by
    \[
    (12)\mapsto\begin{pmatrix}
    0&1&0\\
    1&0&0\\
    0&0&1
    \end{pmatrix},\quad
    (123)\mapsto\begin{pmatrix}
    0&0&1\\
    1&0&0\\
    0&1&0
    \end{pmatrix}.
    \]
    Then $\rho_g$ is unitary for all $g\in G$ (because $\rho_{(12)}$ and $\rho_{(123)}$ are both
    unitary). Moreover,
    \[
    S=\left\langle \begin{pmatrix}
    1\\1\\1
    \end{pmatrix}
    \right\rangle,
    \quad
    T=S^{\perp}=\left\langle
    \begin{pmatrix}
    -1\\1\\0
    \end{pmatrix},
    \begin{pmatrix}
    0\\-1\\1
    \end{pmatrix}
    \right\rangle,
    \]
    are irreducible invariant subspaces of $V=\C^3$. A direct calculation shows that
    in the orthogonal basis $\left\{\begin{pmatrix}
    1\\1\\1
    \end{pmatrix},
    \begin{pmatrix}
    -1\\1\\0
    \end{pmatrix},
    \begin{pmatrix}
    0\\-1\\1
    \end{pmatrix}
    \right\}$
    the matrices $\rho_{(12)}$ and $\rho_{(123)}$ can be written as
    \[
    \rho_{(12)}=\begin{pmatrix}
        1&0&0\\
        0&-1&1\\
        0&0&1
    \end{pmatrix},
    \quad
    \rho_{(123)}=
    \begin{pmatrix}
        1&0&0\\
        0&0&-1\\
        0&1&-1
    \end{pmatrix}.
    \]
\end{example}

The \emph{commutator subgroup} of a group $G$ is defined as the subgroup $[G,G]$ 
generated by the commutators $[x,y]=xyx^{-1}y^{-1}$, that is  
 \[
        [G,G]=\langle[x,y]: x,y\in G\rangle.
\]

Routine calculations show that $[G,G]$ is always a normal subgroup of $G$. 
For example, the commutator subgroup of $\Z$ is the trivial subgroup $\{0\}$, 
as 
\[
[x,y]=x+y-x-y=0
\]
for all $x,y\in\Z$. As an exercise, one can also prove that 
$[\Sym_3,\Sym_3]=\{\id,(123),(132)\}$.

\begin{exercise}
\label{xca:commutator}
Let $H$ be a normal subgroup of $G$. Prove that
$G/H$ is abelian if and only if $[G,G]\subseteq H$.
\end{exercise}

The previous exercise demonstrates, in particular, that for any group 
$G$, the quotient $G/[G,G]$ is always an abelian group. 
This quotient is known as the \emph{abelianization} of 
$G$. 

\begin{exercise}
Let $G$ be a finite group.
Prove that there is a bijection between degree-one representations of $G$ and
degree-one representations of $G/[G,G]$.
\end{exercise}

The following result is simple and crucial. 

\begin{lemma}[Schur]
\index{Schur's!lemma}
    Let $\rho\colon G\to\GL(V)$ and $\psi\colon G\to\GL(W)$ be irreducible representations. If 
    $T\colon V\to W$ is a non-zero invariant map, then $T$ is bijective.  
\end{lemma}

\begin{proof}
    Since $T$ is non-zero and $\ker T$ is an invariant subspace of $V$, it follows that $\ker T=\{0\}$, as $\rho$ is irreducible. Thus 
    $T$ is injective. Since $T(V)$ is a non-zero invariant subspace of $W$ and $\psi$ is irreducible, 
    it follows that $T$ is surjective. Therefore $T$ 
    is bijective.  
\end{proof}

Two applications:

\begin{proposition}
\label{pro:Schur_consequence}
    If $G$ is finite, $\rho\colon G\to\GL(V)$ is an irreducible representation, and $T\colon V\to V$ is invariant, then 
    $T=\lambda\id$ for some $\lambda\in\C$. 
\end{proposition}

\begin{proof}
    Let $\lambda$ be an eigenvalue of $T$. Then $T-\lambda\id$ is invariant, as 
    \[
    (T-\lambda\id)\rho_g=T\rho_g-\lambda\rho_g=\rho_g(T-\lambda\id)
    \]
    for all $g\in G$ since $T$ is invariant. By definition, 
    $T-\lambda\id$ is not bijective. Thus $T-\lambda\id=0$ by Schur's lemma.
\end{proof}

\begin{proposition}
    Let $G$ be a finite abelian group. 
    If $\rho\colon G\to\GL(V)$ is an irreducible representation, then
    $\dim V=1$. 
\end{proposition}

\begin{proof}
    Let $h\in G$. Note that since $G$ is abelian, $T=\rho_h$ is invariant:
    \[
    T\rho_g=\rho_h\rho_g=\rho_{hg}=\rho_{gh}=\rho_g\rho_h=\rho_gT.
    \]
    By the previous proposition, 
    there exists $\lambda_h\in\C$ such that $\rho_h=\lambda_h\id$. If $v\in V\setminus\{0\}$, 
    then $V=\langle v\rangle$. In fact, since 
    $\langle v\rangle$ is a non-zero invariant subspace of $V$ and $\rho$ is irreducible, 
    it follows that $V=\langle v\rangle$. 
\end{proof}

